\subsection{Notas Das Preventivas Planejadas De Máquinas Críticas}

\begin{minted}{sql}
    SELECT * FROM Nota n 
        INNER JOIN PlanejamentoPreventiva pp ON pp.id_nota = n.id 
        INNER JOIN Preventiva p ON p.id = pp.id_preventiva
        INNER JOIN Maquina m ON m.id = p.id_maquina
    WHERE m.critico = 1;
\end{minted}
\begin{tabularx}{1\textwidth} {
        | >{\raggedright\arraybackslash}X
        | >{\centering\arraybackslash}X
        | >{\raggedleft\arraybackslash}X |}
    \hline
    \multicolumn{2}{|c|}{Resultado}  \\
    \hline
    \textbf{id} & \textbf{descrição} \\
    \hline
    2           & Alta Temperatura   \\
    \hline
\end{tabularx}

\vspace{1cm}

\subsection{Consumo Total De Cada Material Em Máquinas Críticas}

\begin{minted}{sql}
    SELECT m2.descrição , SUM(cm.quantidade) as total FROM ConsumoMaterial cm 
        INNER JOIN Maquina m ON m.id = cm.id_maquina
        INNER JOIN Material m2  ON m2.id = cm.id_material 
    WHERE m.critico = 1
    GROUP BY cm.id_material;
\end{minted}
\begin{tabularx}{1\textwidth} {
        | >{\raggedright\arraybackslash}X
        | >{\centering\arraybackslash}X
        | >{\raggedleft\arraybackslash}X |}
    \hline
    \multicolumn{2}{|c|}{Resultado}     \\
    \hline
    \textbf{descrição} & \textbf{total} \\
    \hline
    Manta Cinza        & 8              \\
    \hline
    Bomba Hidraulica   & 1              \\
    \hline
    Cabo Conexão       & 1              \\
    \hline
    Valvula Pneumatica & 2              \\
    \hline
\end{tabularx}

\vspace{1cm}

\subsection{Pegar Os Nomes Dos Compradores Livres (Acompanhando Nenhuma Compra)}

\begin{minted}{sql}
    SELECT u.nome  FROM Comprador c 
        LEFT JOIN PedidoCompra pc ON pc.id_comprador = c.id
        INNER JOIN Usuário u ON u.chapa = c.chapa 
    WHERE pc.id IS NULL;
\end{minted}
\begin{tabularx}{1\textwidth} {
        | >{\raggedright\arraybackslash}X
        | >{\centering\arraybackslash}X
        | >{\raggedleft\arraybackslash}X |}
    \hline
    \multicolumn{1}{|c|}{Resultado} \\
    \hline
    \textbf{nome}                   \\
    \hline
    Enzo                            \\
    \hline
\end{tabularx}

\pagebreak

\subsection{Listar As Máquinas Críticas E A Quantidade De Ordens E Notas}

\begin{minted}{sql}
	SELECT m.id, m.descrição, m.critico, 
            COUNT(n.id) - COUNT(o.id) as notas, COUNT(o.id) as ordens 
        FROM Maquina m 
	    LEFT JOIN Nota n ON n.id_maquina = m.id
	    LEFT JOIN Ordem o ON o.id_nota = n.id
	GROUP BY m.id
	ORDER BY (COUNT(n.id)) DESC, m.descrição ASC;
\end{minted}
\begin{tabularx}{1\textwidth} {
        | >{\raggedright\arraybackslash}X
        | >{\centering\arraybackslash}X
        | >{\centering\arraybackslash}X
        | >{\centering\arraybackslash}X
        | >{\centering\arraybackslash}X
        | >{\raggedleft\arraybackslash}X |}
    \hline
    \multicolumn{5}{|c|}{Resultado}                                                                  \\
    \hline
    \textbf{id} & \textbf{descrição}           & \textbf{critico} & \textbf{notas} & \textbf{ordens} \\
    \hline
    5           & Tamboreamento                & 0                & 2              & 1               \\
    \hline
    2           & Centro de Tratamento         & 1                & 1              & 1               \\
    \hline
    4           & Máquina Jateamento           & 0                & 1              & 1               \\
    \hline
    3           & Retificadora                 & 1                & 1              & 1               \\
    \hline
    6           & Prensa de Bancada Hidráulica & 1                & 0              & 1               \\
    \hline
    1           & Centro de Torneamento        & 1                & 0              & 0               \\

    \hline
\end{tabularx}

\vspace{1cm}

\subsection{Listar Os Materiais Com Seu Fornecedor Que Vende A Menor Preço}

\begin{minted}{sql}
    SELECT m.descrição, fm.preço, f.nome FROM Material m 
        INNER JOIN FornecimentoMaterial fm
        ON fm.id_material = m.id AND fm.preço = (SELECT MIN(fm2.preço) 
                                                 FROM FornecimentoMaterial fm2 
                                                 WHERE fm2.id_material = m.id)
        INNER JOIN Fornecedor f ON f.id = fm.id_fornecedor 											 
\end{minted}
\begin{tabularx}{1\textwidth} {
        | >{\raggedright\arraybackslash}X
        | >{\centering\arraybackslash}X
        | >{\centering\arraybackslash}X
        | >{\centering\arraybackslash}X
        | >{\raggedleft\arraybackslash}X |}
    \hline
    \multicolumn{3}{|c|}{Resultado}                           \\
    \hline
    \textbf{descrição} & \textbf{fornecedor} & \textbf{preço} \\
    \hline
    Modulo de Potencia & 32.0                & NIKKEYPAR      \\
    \hline
    Manta Cinza        & 43.0                & SIEMENS        \\
    \hline
    Bomba Hidraulica   & 87.0                & BOSCH          \\
    \hline
    Cabo Conexão       & 39.0                & THEVAL         \\
    \hline
    Filtro             & 10.0                & NORTEL         \\
    \hline
    Lampada            & 100.0               & NORTEL         \\
    \hline
    Valvula Pneumatica & 123.0               & NIKKEYPAR      \\
    \hline
\end{tabularx}

\pagebreak

\subsection{Listar Todos Os Materiais Requisitados Para Preventivas Planejadas
    \\ Contendo Os Dados Da Compra, Preço, Preventiva, Máquina, Material}

\begin{minted}{sql}
    SELECT u.nome as comprador,
        m2.descrição as maquina,
        p.descrição as preventiva,
        CONCAT(p.periodicidade, ' : ', p.meses) as periodos,
        m.descrição as material,
        f.nome as fornecedor,
        ((fm.preço * r.quantidade) + pc.frete) as preço
    FROM Requisição r
        INNER JOIN Material m ON m.id = r.id_material
        INNER JOIN PedidoCompra pc ON pc.id = r.id_compra
        INNER JOIN Fornecedor f ON f.id = pc.id_fornecedor
        INNER JOIN FornecimentoMaterial fm ON fm.id_material = r.id_material
        AND fm.id_fornecedor = f.id
        INNER JOIN Ordem o ON o.id = r.id_ordem
        INNER JOIN Nota n ON o.id_nota = n.id
        INNER JOIN PlanejamentoPreventiva pp ON pp.id_nota = n.id
        INNER JOIN Preventiva p ON p.id = pp.id_preventiva
        INNER JOIN Maquina m2 ON m2.id = n.id_maquina
        INNER JOIN Comprador c ON pc.id_comprador = c.id
        INNER JOIN Usuário u ON u.chapa = c.chapa;
\end{minted}
\begin{tabularx}{1\textwidth} {
        | >{\raggedright\arraybackslash}X
        | >{\centering\arraybackslash}X
        | >{\centering\arraybackslash}X
        | >{\centering\arraybackslash}X
        | >{\centering\arraybackslash}X
        | >{\centering\arraybackslash}X
        | >{\raggedleft\arraybackslash}X |}
    \hline
    \multicolumn{7}{|c|}{Resultado}                                                                                                                   \\
    \hline
    \textbf{comprador} & \textbf{maquina}     & \textbf{preventiva}   & \textbf{periodos} & \textbf{material}  & \textbf{fornecedor} & \textbf{preço} \\
    \hline
    Erika              & Centro de Tratamento & Prevenção de Incêndio & SM : Jan,Jun      & Modulo de Potencia & SIEMENS             & 410.0          \\
    \hline
    Erika              & Tambore- amento      & Troca de Motor        & AN : Fev          & Manta Cinza        & SIEMENS             & 536.0          \\
    \hline
    John               & Tambore- amento      & Troca de Motor        & AN : Fev          & Bomba Hidraulica   & BOSCH               & 726.0          \\
    \hline
\end{tabularx}


\pagebreak


\subsection{Listar Todos Os Relacionamento Entre Máquina e Material}

\begin{minted}{sql}
    SELECT m3.descrição as maquina,
        m4.descrição as material,
        m4.classe
    FROM (
            (
                SELECT cm.id_material, cm.id_maquina
                FROM ConsumoMaterial cm
            )
            UNION
            (
                SELECT r.id_material, m2.id as id_maquina
                FROM Requisição r
                    INNER JOIN Ordem o ON o.id = r.id_ordem
                    INNER JOIN Nota n ON n.id = o.id_nota
                    INNER JOIN Maquina m2 ON m2.id = n.id_maquina
            )
        ) DataTable
        INNER JOIN Maquina m3 ON m3.id = DataTable.id_maquina
        INNER JOIN Material m4 ON m4.id = DataTable.id_material
    ORDER BY m3.descrição ASC,
        m4.descrição ASC;
\end{minted}
\begin{tabularx}{1\textwidth} {
        | >{\raggedright\arraybackslash}X
        | >{\centering\arraybackslash}X
        | >{\centering\arraybackslash}X
        | >{\centering\arraybackslash}X
        | >{\raggedleft\arraybackslash}X |}
    \hline
    \multicolumn{3}{|c|}{Resultado}                                   \\
    \hline
    \textbf{maquina}             & \textbf{material}  & \textbf{tipo} \\
    \hline
    Centro de Torneamento        & Bomba Hidraulica   & Preventivo    \\
    \hline
    Centro de Torneamento        & Cabo Conexão       & Corretivo     \\
    \hline
    Centro de Tratamento         & Modulo de Potencia & Preventivo    \\
    \hline
    Máquina Jateamento           & Bomba Hidraulica   & Preventivo    \\
    \hline
    Máquina Jateamento           & Cabo Conexão       & Corretivo     \\
    \hline
    Máquina Jateamento           & Lampada            & Consumo       \\
    \hline
    Máquina Jateamento           & Modulo de Potencia & Preventivo    \\
    \hline
    Prensa de Bancada Hidráulica & Cabo Conexão       & Corretivo     \\
    \hline
    Prensa de Bancada Hidráulica & Filtro             & Consumo       \\
    \hline
    Prensa de Bancada Hidráulica & Manta Cinza        & Consumo       \\
    \hline
    Prensa de Bancada Hidráulica & Valvula Pneumatica & Corretivo     \\
    \hline
    Retificadora                 & Lampada            & Consumo       \\
    \hline
    Retificadora                 & Modulo de Potencia & Preventivo    \\
    \hline
    Retificadora                 & Valvula Pneumatica & Corretivo     \\
    \hline
    Tamboreamento                & Bomba Hidraulica   & Preventivo    \\
    \hline
    Tamboreamento                & Manta Cinza        & Consumo       \\
    \hline
    Tamboreamento                & Motor              & Preventivo    \\
    \hline
\end{tabularx}

